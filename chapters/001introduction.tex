\chapter{Einleitung}
\label{chap:intro}

\begingroup
    \leftskip=4em
    \rightskip\leftskip{}
    \textit{``Man kann nicht nicht kommunizieren, denn jede Kommunikation (nicht nur mit Worten) ist Verhalten und genauso wie man sich nicht nicht verhalten kann, kann man nicht nicht kommunizieren.''}~\cite{watzlawick2011menschliche}
    \par
\endgroup

Diese Projektarbeit ist Teil eines Projektes im Rahmen des Moduls \textit{BZG3101: Kommunikation 1 Deutsch} der Berner Fachhochschule für angewandte Wissenschaft.

Das Ziel dieser Projektarbeit ist die Erstellung und Analyse eines kommunikativen Profils sowie die Analyse der direkten und indirekten unternehmensinternen Kommunikation von Sven Osterwalder.

Im ersten Teil der Arbeit erfolgt eine Erstellung und Analyse des persönlichen kommunikativen Profils. Im zweiten Teil der Arbeit wird die unternehmensinterne Kommunikation beobachtet und analysiert. Im letzten Teil der Arbeit erfolgt eine Diskussion der gewonnenen Erkenntnisse.

%  Die Einleitung führt die Leserinnen und Leser in die Thematik der Arbeit ein. Sie setzt den Rahmen für die Arbeit, skizziert den aktuellen Wissensstand und geht auf die konkreten Fragestellungen und Ziele der vorliegenden Arbeit ein. Aus der Einleitung muss klar hervorgehen, weshalb die Arbeit gemacht wurde und welche Bedeutung ihr zukommt.
%  
%  \begin{itemize}
%      \item Ziel, worum geht es?
%      \item Eine gute Einleitung motiviert, den ganzen Bericht zu lesen.
%      \item Eine gute Einleitung enthält alle später angesprochenen Themen, aber nur diese!
%      \item Was will ich wissen?
%      \item Was habe ich für Daten? (Wie gesammelt, woher, Weiterarbeit?)
%      \item Welche Modelle brauche ich? (Matlab Toolbox)
%      \item Was erwarte ich von den Resultaten?
%  \end{itemize}
